\documentclass[11pt,a4paper]{article}
\usepackage[latin1]{inputenc}
\usepackage[pdftex]{graphicx,color}  
\usepackage[linkcolor=blue, colorlinks=true]{hyperref}
\usepackage{multicol}
\usepackage{titlesec}

\setlength{\hoffset}{0pt}
\setlength{\voffset}{0pt}
\setlength{\oddsidemargin}{0pt}
\setlength{\topmargin}{0pt}
\setlength{\headheight}{0pt}
\setlength{\headsep}{0pt}
\setlength{\textheight}{\paperheight}
\addtolength{\textheight}{-2.2in}
\setlength{\parindent}{0cm}
\setlength{\parskip}{.2cm}
\setlength{\tabcolsep}{.2cm}
\setlength{\textwidth}{\paperwidth}
\addtolength{\textwidth}{-2in}
\clubpenalty = 10000
\widowpenalty = 10000 
\displaywidowpenalty = 10000

\titlespacing\section{0pt}{6pt plus 2pt minus 1pt}{1pt plus 1pt minus 1pt}

\titleformat{\section}{\normalfont\fontsize{12}{15}\bfseries}{\thesection}{1em}{}

% XXX also shrink section sizes slightly

\pdfinfo {            
	/Title(Stable, Composable and Efficient Paging of Result Sets)
}

\title{Stable, Composable and Efficient Paging\\ of Result Sets}
\author{Robert Lemmen $<$robertle@semistable.com$>$}
\date{\today}

\begin{document}
\maketitle
\bigskip

\begin{abstract}
\noindent Paging of results sets, showing a limited section of the results with
the ability to navigate to the adjacent sections, is a common and seemingly
trivial method used in computer systems. As usual, the devil is in the details
however. A simple and widespread approach is to use and offset into the result
set. This text shows problems resulting from this strategy, how they can be
overcome with a small modification, and how this can be leveraged to solve
even more interesting problems.
\end{abstract}
\vspace{40pt}

\begin{multicols}{2}
\section*{A Paging Example}
In order to illustrate the next sections, it is useful to have a common and
simple example case. Suppose we have a computer system containing a set of
records and making these records available to the user. We only want to present
a limited set of records with a given size to the user at a time, and the user 
should be able to retrieve the next set, or go back to the previous set. A very
common setup that we are all familiar with.

In our example the records will initially only be simple numbers. Not a very
useful system, but sufficient for this discussion. Our system will contain only
the numbers 2, 4, 5, 6, 8, 9, 11, 12, 14, 16, 17, 18, 19, 21, 24, 26. We will
also fix the size of a result set page to 5. We will allow a set of retrieval
options like filtering, search and ordering as we expand the example, but
initially the number are always presented in their natural numerical order, and
all numbers are returned. 

So the if the user submits a query, the first page presented should contain the
numbers 2, 4, 5, 6, and 8, and the next page 9, 11, 12, 14, 16. Overall there
will be four pages, with the last on consisting of only one element, 26. 

\section*{Using Offsets}
The most common approach to implementing this is using an offset into full
result set. So for example a function would be invoked with query parameters
like ordering and filtering predicates, and an initial offset = 0 and a
pagesize of 5. The implementation of this function would somehow apply the
ordering and filtering, and then returnd the items starting with the element at
position {\em offset} in the list, stopping after it has returned {\em
pagesize}. 

When the user navigates to the next page, the presentation logic simply adds the
pagesize to the current offset, yielding the offset of the next page and uses
this for the next query. In the same way, the previous page can be addressed by
subtracting the pagesize from the current offset. 

\section*{Stability}
It is interesting to observe what happens when we modify the data in our system
between two calls that present adjacent pages. There is no ``transactionality''
over the two page presentations, so we cannot expect to see an isolated full
view of either the old or the new state. In other words if an item gets added to
the list then we might or might not see it when paging through the result set,
depending on when it was added relative to the flip from one page to the next,
and depending on which page it is on. 

What we do not expect however, is that the addition or removal of an item has an
effect on the visibility of other items. Unfortunately this is the case however. 

In our example we have just retrieved the first page (2, 4, 5, 6, 8), when the
number 6 gets deleted from the system. If we were to re-retrieve the first page
at this point, we would get 2, 4, 5, 8 and 9 instead. But we do retrieve the
second page, unaware of the modification, and get 11, 12, 14, 16, and 17. So the
removal of item 6 has hidden led to the fact that we have seen 6, but have
missed 9 instead. In a similar effect, the addition of items in a page before
the one we are currently looking at will lead to items being present on multiple
pages. This effect gets worse with the size of the result set, as addition or
removal of items affects all following pages. The bigger the result set, the
more pages are affected by a change.

\section*{Efficiency}
Earlier we said the implementation of our query method would return items 
``...starting with the element at position {\em offset}...''. This is worth
examining: while some data structures that could potentially be used to hold our
data do support direct and efficient addressing of elements by their position
(e.g. arrays), these are unlikely to be used for any real-world application.
Much more likely are search trees of some sort or other data structure that 
allow efficient ($\mathcal{O}(n\log{}n)$ or better) search and insertion. In 
these, the only way to locate an element by a
given position is to iterate from the start to that position and simply discard
all elements prior to the desired one, an $\mathcal{O}(n)$ and search. There are
no data structures that allow all three operations of insertion, search and
direct addressing by position efficiently. 

Even if there was a data structure that would support this, it could not
continue to work in the presence of a filtering predicate. If for example we
instruct our system to only return even numbers, it would have to iterate
through the items in our database, check for each of them whether it is even or
not, and keep count of how many fulfill this criteria. 

The fact that finding a single page has a complexity of $\mathcal{O}(n)$ is bad
enough, but it also means that traversing all pages has a complexity of
$\mathcal{O}(n^2)$, always a sign of trouble.

% XXX may be hidden by database implementation

\section*{Simple Locators}

% XXX explain simple case
% XXX stability
% XXX efficiency
% XXX paging backwards

% XXX more stuff...


\end{multicols}
\end{document}
